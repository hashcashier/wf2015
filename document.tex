\documentclass[twocolumn]{article}

\usepackage{listings}
\usepackage{graphicx}
\usepackage{color}
\usepackage{fancyhdr}
\usepackage{courier}
\usepackage[margin=2cm]{geometry}
\usepackage{caption}
\usepackage{amsmath, amssymb, amsfonts}
\usepackage{comment, fancyvrb, fancyhdr}
\usepackage{color}
\usepackage{cmap}
\usepackage{array}


\definecolor{mygreen}{rgb}{0,0.6,0}
\definecolor{mygray}{rgb}{0.5,0.5,0.5}
\definecolor{mymauve}{rgb}{0.58,0,0.82}
\definecolor{lightgray}{rgb}{0.9,0.9,0.9}
\definecolor{darkred}{rgb}{0.55, 0.0, 0.0}
\definecolor{burntorange}{rgb}{0.8, 0.33, 0.0}

\lstset{
	language=C++,
         	basicstyle=\scriptsize\ttfamily,
         	backgroundcolor=\color{lightgray},
         	numberstyle=\tiny,          	% Stil der Zeilennummern
         	keywordstyle=\color{darkred}\bfseries,
         	commentstyle=\color{mygreen}\bfseries,
        	numbers=left,               	% Ort der Zeilennummern
        	 %stepnumber=2,               % Abstand zwischen den Zeilennummern
         	numbersep=5pt,              	% Abstand der Nummern zum Text
         	tabsize=2,                  	% Groesse von Tabs
         	extendedchars=true,         %
         	breaklines=true,            	% Zeilen werden Umgebrochen
         	breakatwhitespace=false,
         	frame=none,
         	stringstyle=\color{blue}\ttfamily, 	% Farbe der String
         	showspaces=false,           		% Leerzeichen anzeigen ?
         	showtabs=false,             		% Tabs anzeigen ?
         	showstringspaces=false,      	% Leerzeichen in Strings anzeigen ?        
         	xleftmargin=0pt,
	abovecaptionskip=0pt,
	belowcaptionskip=0pt
 }

 \lstloadlanguages{
         C++,
         Java
 }

\DeclareCaptionFont{blue}{
	\bfseries
	\color{burntorange}
}
\DeclareCaptionFont{black}{
	\color{black}
}

\captionsetup[lstlisting]{
	singlelinecheck=false,
	labelfont={blue},
	textfont={black}
}


\setlength{\parindent}{0pt}
%\setlength{\parskip}{3mm}

\renewcommand{\headrulewidth}{0pt}
\renewcommand{\footrulewidth}{0pt}

\newcommand{\Topic}[1]{\textbf{#1}}

\pagestyle{fancy}

\fancyhead[L]{The German University in Cairo}
\fancyhead[C]{\textbf{Bool Shift!}}
\fancyfoot[C]{KH MB RK}
\fancyhead[R]{\thepage}

\begin{document}
	\tableofcontents
	\newpage
	\begin{tabular}{| l || m{7cm} | }
		\hline
		\textbf{Problem} & \textbf{Tags}\\	\hline
		01 A & \\	\hline
		02 B & \\	\hline
		03 C & \\	\hline
		04 D & \\	\hline
		05 E & \\	\hline
		06 F & \\	\hline
		07 G & \\	\hline
		08 H & \\	\hline
		09 I & \\	\hline
		10 J & \\	\hline
		11 K & \\	\hline
		12 L & \\	\hline
		13 M & \\	\hline
	\end{tabular}

	\begin{tabular}{| l | l || m{12mm}| }
		\hline
		\textbf{Time} & \textbf{Meeting Description} & \textbf{Check} \\	\hline
		030 & All Problems Read. Write Tags. & 	\\	\hline
		060 & Ace Decided. Choose Coder.	&	\\	\hline
		120 & Decide \& Order Solveable Problems &	\\	\hline
		150 & Status Check &	\\	\hline
		180 & Status Check &	\\	\hline
		210 & Status Check &	\\	\hline
		240 & Status Check &	\\	\hline
		270 & Status Check &	\\	\hline
	\end{tabular}
	\newpage

	\section{2D Geometry}
		\lstinputlisting[label=template,caption=Primitives]{cpp/2dprimitives.cpp}
		\lstinputlisting[label=template,caption=Triangulation]{cpp/triangulation.cpp}
	\section{3D Geometry}
	\section{Combinatorics}
		\lstinputlisting[label=template,caption=Basics]{cpp/combinatorics.cpp}
	\section{Data Structures}
	\section{Graph Theory}
	\section{Number Theory}
		\lstinputlisting[label=template,caption=Gaussian Elimination]{cpp/gaussian_elimination.cpp}
		\lstinputlisting[label=template,caption=Tortoise \& Hare]{cpp/tortoise_hare.cpp}
	\section{Search}
		\lstinputlisting[label=template,caption=Ternary Search]{cpp/ternary_search.cpp}
	\section{Strings}
\end{document}